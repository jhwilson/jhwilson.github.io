\documentclass{jhwhw}

\author{PHYS 7221}
\title{Homework \#6}
\date{Due: 11 October 2022}

\begin{document}

\problem{[20 pts] Escaping particles constituents}

A particle with internal energy $E_{\mathrm{int}}$ is traveling with velocity $\mathbf V$ perpendicular to a screen.
At a time before it hits the screen, the particle disintegrates losing energy $\epsilon$ into kinetic energy, the two components $m_{1}$ and $m_{2}$ are released isotropically in the center of mass frame (while maintaining momentum conservation).
What is the fraction of experimental runs in which particles $m_{1}$ hits the screen as a function of $|\mathbf V|$?
What is the fraction of experimental runs in which they both hit the screen as a function of $|\mathbf V|$?

\problem{[20 pts] Linear Triatomic molecule}

In class we discussed the triatomic molecule. Starting with that solution consider the following
\begin{enumerate}
  \item Find the solution $y_{i}(t)$ for $i=1,2,3$ with initial data $y_{1}(0) = -A$, $y_{2}(0) = Am/M$, $y_{3}(0) = 0$, and $\dot y_{1} = \dot y_{2} = \dot y_{3} = 0$.
  \item Describe the dynamics of the center of mass of the system for this solution.
  \item Define $\omega \equiv \frac{\omega_{1} + \omega_{2}}2$ and $\Delta \omega = \frac{\omega_{1} - \omega_{2}}2$ where $\omega_{1} = \sqrt{k/m}$ and $\omega_{2} = \sqrt{k (2M +m)/(mM)}$. Describe the dynamics of each particle for the solution obtained above in the limit $\omega1 \approx \omega_{2}$, i.e., $\omega \gg \Delta \omega$ (the phenomenon of ``beats'' appears).
\end{enumerate}

\emph{Hint: } Use $\cos(\omega_{1} t) \pm \cos(\omega_{2} t) = (1 \pm 1) \cos(\omega t) \cos(\Delta \omega t) - (1 \mp 1) \sin(\omega t) \sin(\Delta \omega t)$.

\problem{[20 pts] Two masses attached to walls}
  Two identical bodies of mass $m$ are attached by identical springs of spring constant $k$ as shown in the figure.
\begin{enumerate}
  \item Find the frequencies of small oscillation of this system (normal frequencies).
  \item The first mass (reading left-to-right) is displaced from its position by a small distance $a_{1}$ to the right while the second mass is not moved from its position. If the two masses are released with zero velocity, what is the subsequent motion of the second mass?
\end{enumerate}

\begin{center}
  \includegraphics{Plot1.pdf}
\end{center}

\problem{[20 pts] Three masses on a circle}

Three equal mass particles have a stationary position at the vertices of an equilateral triangle on a circle of radius $R$.
They are connected by springs with spring constants $k$ that lie along the arcs of the circumference of the circle.
The particles and the springs are constrained to only lie on the circle so that the potential energy of the spring is determined solely by the arc length it covers.
\begin{enumerate}
  \item Determine the normal frequencies and normal modes in the plane. Describe the physical effects of any zero frequency mode.
  \item Suppose we change one spring constant by $\delta k$ while we keep the other two unchanged. Compute the change in the normal frequencies and normal modes at first order in $\delta k$.
  \item Suppose we now change the mass of one particle by an amount $\delta m$. What is the change in the normal frequencies and modes?
\end{enumerate}

\end{document}
